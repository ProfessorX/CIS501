
%%% Local Variables: 
%%% mode: latex
%%% TeX-master: t
%%% End: 
\documentclass{article}


% Packages
% \usepackage{algorithm}
\usepackage{algorithmic}



\begin{document}
I would like to describe in Chinese. But after months of training, I
find English more comfortable.
\par
This time let's work together with heart and soul, to solve this
problem and win the happiness of Emirati. You are the best.
\begin{enumerate}
\item Load training data chunk 1--5 into MATLAB.\@ After this step,
  they should be cell array.
\item Missing categorical values. They should be represented as
  ``NaN'' in MATLAB. Replace missing \textbf{categorical} values with
  \textbf{0}. 

\item Missing numeric values. They should be represented as NaN in
  MATLAB.\@ Replace them with mean of that column. (I suppose it would
  be more clear to describe in Algorithms. The rest is done with
  algorithms.)

\end{enumerate}


\begin{figure}[htbp]
  \centering
  \begin{algorithmic}[1]
    \FOR{each column}
    \STATE~Check if there is $NaN$.
    \STATE~Add a new \textbf{row}. If $Column(j)~has~NaN$,
    $row(i,j)=1$. Else mark $0$.
    \ENDFOR
  \end{algorithmic}
  \caption{Missing values}
\end{figure}



\begin{figure}[htbp]
  \centering


  \begin{algorithmic}[1]
    \FOR{Missing categorical values}
    \STATE~Replace $NaN$ in missing categorical values with $0$. 
    \STATE~if elements in categorical values$ \equiv NaN$
    \STATE~$NaN --> 0$
    \ENDFOR
  \caption{Missing categorical values}
  \end{algorithmic}
\end{figure}

\begin{figure}[htbp]
  \centering
  \begin{algorithmic}[1]
    \STATE~Compute each column's mean. $mean(Column)$.
    \STATE~Replace $NaN$ in numerical columns with each column's mean.
  \end{algorithmic}


  \caption{Missing numeric values}
\end{figure}


\begin{figure}[htbp]
  \centering
  I will describe this in Chinese, in a separate file.
  \caption{Recode categorical values}
\end{figure}


\begin{figure}[htbp]
  \centering
  \begin{algorithmic}[1]
    \STATE~use $abs(mas())$ to calculate each column's max value.
    \STATE~Normalize each column. For each value i \textbf{in column
      j}, replace value i with $i / max~value~in~that~column$. Namely,
    scale the value to [0--1].
  \end{algorithmic}
  \caption{Normalize}
\end{figure}

\begin{figure}[htbp]
  \centering
Sample code is given as follows. I will describe in a separate file in
Chinese. 
\begin{verbatim}.
matlab> SPECTF = csvread('SPECTF.train');
 % read a csv file
matlab> labels = SPECTF(:, 1);
 % labels from the 1st column
matlab> features = SPECTF(:, 2:end); 
matlab> features_sparse = sparse(features); 
% features must be in a sparse matrix
\end{verbatim}
  \caption{Write to csv}
\end{figure}
\end{document}
